\documentclass[fleqn]{report}

\input{latex_files/preamble_dark}
\input{latex_files/macros}
\input{latex_files/letterfonts}

\title{formulas used by Pi-chuck}
\author{Paul TAVENEAUX}
\date{\today}

\usepackage{cancel}
\usepackage[svgnames]{xcolor}

\newcommand\Ccancel[2][black]{\renewcommand\CancelColor{\color{#1}}\cancel{#2}}

\pagecolor{black}
\color{white}

\begin{document}
\maketitle
\newpage

\chapter{Basics}

\section{Original formula}
\[
  \frac{1}{\pi} = \frac{1}{426880\sqrt{10005}}\sum_{ k = 0 }^{\infty}\frac{{(-1)}^k(6k)!(13591409 + 545140134k)}{(3k)!{(k!)}^{3}640320^{3k}}
\]

\section{$\pi$ in terms of $a$ and $b$}

\begin{align*}
  \frac{1}{\pi} & = \frac{1}{426880\sqrt{10005}}\left[13591409\sum_{k = 0}^{\infty}\frac{{(-1)}^k(6k)!}{(3k)!{(k!)}^{3}640320^{3k}} + 545140134\sum_{k = 0}^{\infty}\frac{{(-1)}^k(6k)!k}{(3k)!{(k!)}^{3}640320^{3k}}\right] \\
  \pi           & = \frac{426880\sqrt{10005}}{1359140a + 545140134b}                                                                                                                                                         \\
\end{align*}
with
\begin{align*}
  a                  & = \sum_{k = 0}^{\infty}\frac{{(-1)}^k(6k)! }{(3k)!{(k!)}^{3}640320^{3k}} \\
  \text{and} \quad b & = \sum_{k = 0}^{\infty}\frac{{(-1)}^k(6k)!k}{(3k)!{(k!)}^{3}640320^{3k}}
\end{align*}

\chapter{computing $ 426880\sqrt{10005} $}

Remark that $ 426880 \sqrt{10005} $ is the positive root of $ f(x) = x^2
  -426880^2 \cdot 10005$
\pf{}{
	\begin{alignat*}{3}
    &&x ^2 - 426880 ^2 \cdot 10005 & = 0                                \\
    &\iff& x ^2                    & = 426880 ^2 \cdot 10005            \\
    &\iff& x                       & = \pm \sqrt{426880 ^2 \cdot 10005} \\
    &\iff& x                       & = \pm 426880 \sqrt{10005}
  \end{alignat*}
}

\section{Newton's method}
let $a \in \R$ such that $\forall x \in \R,\ f(x) = x^2 - a$\\
the roots of $f$ are quite clearly $\sqrt{a} $ and $-\sqrt{a} $\\
let $\alpha$ be the positive root of  $f$: $\alpha = \sqrt{a}$ 
It follows from $f$'s definition that:
\begin{align*}
	 &\forall x \in \R,\ f'(x) = 2x\\
	 &\forall x \in \R,\ f''(x) = 2 \\
\end{align*}


\noindent
Observe that:
\begin{align*}
	\forall (\alpha, x_n) \in \R^2 ,\ x_n^2 - a + 2x_n(\alpha - x_n) + \frac{1}{2}2(\alpha - x_n)^2 &= x_n^2 - a + 2\alpha x_n - 2x_n^2 + \alpha ^2 - 2\alpha x_n + x_n^2 \\
	&= x_n^2 - a + 2\alpha x_n - 2x_n^2 + \alpha ^2 - 2\alpha x_n + x_n^2\\
	&= \Ccancel[white]{x_n^2 + x_n^2 - 2x_n^2} - a + \alpha ^2 + \Ccancel[white]{2\alpha x_n - 2\alpha x_n} \\
	&= \alpha^2 - a \\
	x_n^2 - a + 2x_n(\alpha - x_n) + \frac{1}{2}2(\alpha - x_n)^2 &= f(\alpha) \tag{1} \\
\end{align*}

Note that:
\begin{align*}
	f'(x) = 0 \iff x = 0 \\
	\implies \forall \epsilon _0 \in ]0; \alpha[,\ x \in [\alpha - \epsilon _0; \alpha + \epsilon _0] \implies x > 0 \implies f'(x) \neq 0 \tag{2}
\end{align*}

\noindent
Let $g \in \R$ be an initial guess of $\alpha$ \\
And let $ {(x_n)}_{n \in \N}$ be a sequence defined by:
\begin{align*}
  \left\{
  \begin{aligned}
    x_0                                & = g \\
    \quad \forall n \in \N,\ x_{n + 1} & = x_n - \frac{f(x_n)}{f'(x_n)}
  \end{aligned}
  \right.
\end{align*}
and let $(\epsilon _n)_{n \in \N}$ be a sequence defined by : $\forall n \in \N,\ \epsilon _n = \alpha - x_n$

\clm{existance of $x_n$}{}{
	\begin{align*}
		g = x_0 \in [\alpha - \epsilon _0; \alpha + \epsilon _0] \implies &\forall n \in \N,\ f'(x_n) \neq 0 \\
		\implies &\forall n \in \N,\ x_n \in \R
	\end{align*}
}

\pf{existance of $x_n$}{}{
	\noindent
	let $\forall n \in \N,\ P_n$ be a propostion s.t. $\forall n \in \N,\ P_n: g = x_0 \in [\alpha - \epsilon _0; \alpha + \epsilon _0] \implies x_n \in [\alpha - \epsilon _0; \alpha + \epsilon _0]$\\
	By induction:\\
	Initialisation: case $n = 0$: $x_0 = g \in [\alpha - \epsilon _0; \alpha + \epsilon _0]$ it immidiatly follows that $x_0 \in [\alpha - \epsilon _0; \alpha + \epsilon _0]$\\
	Induction: \\
	let $n \in \N$ s.t. $P_n$ is verified:
	\begin{alignat*}{3}
		&&g = x_0 \in [\alpha - \epsilon _0; \alpha + \epsilon _0] \implies &x_n \in [\alpha - \epsilon _0; \alpha + \epsilon _0]\\
		&\iff& &\alpha - \epsilon _0 \le x_n \le \alpha + \epsilon _0\\
		&\iff &
	\end{alignat*}
}

\clm{convergence of $(x_n)_{n \in \N}$}{}{
	\[
		g \in ]0; \alpha[ \implies (x_n)_{n \in \N}\ \text{converges to}\ \alpha
	\] 
}

\pf{convergence of $(x_n)_{n \in \N}$}{

}

\chapter{computing the sum}

let ${(a_n)}_{n \in \N}$ be a series such that:
\[
  \forall n \in \N, a_n = \frac{{(-1)}^n(6n)!}{(3n)!{(n!)}^{3}640320^{3n}}
  .\]
and let ${(b_n)}_{n \in \N}$ be a series defined by:
\[
  \forall n  \in \N,\ b_n = n * a_n = \frac{{(-1)}^n(6n)!n}{(3n)!{(n!)n}^{3}640320^{3n}}
  .\]

\section{$a$ and  $ b$  terms of  ${(a_n)}_{n \in \N}$  ${(b_n)}_{n \in \N}$}
It is pretty clear from $ {(a_n)}_{n \in \N}$'s definition that:
\[
  a = \sum_{k=0}^{\infty} a_k
  .\]
also from $ {(b_n)}_{n \in \N}$'s definition:
\[
  b = \sum_{k=0}^{\infty} b_k = \sum_{k=0}^{\infty} k \cdot a_k
  .\]

\section{recursive definition of ${(a_n)}_{n \in \N}$}

\dfn{recursive definition of ${(a_n)}_{n \in \N}$}{
  \begin{align*}
    \left\{
    \begin{aligned}
      a_0                          & = 1                                                                       \\
      \forall n \in \N,\ a_{n + 1} & = -\frac{24(6k + 5)(2k + 1)(6k + 1)}{{(k + 1)}^{3} 640320^{3}} \times a_n \\
    \end{aligned}
    \right.
  \end{align*}
}

\pf{Proof of the recursive definition of ${(a_n)}_{n \in \N}$ }{
  We suppose that $ \forall n \in \N,\ a_n \neq 0$ and compute:
  \begin{align*}
    \frac{a_{n + 1}}{a_n} & = \frac{ \frac{ {(-1)}^{k + 1} (6(k + 1))! }{ (3(k + 1))! {((k + 1)!)}^{3} 640320^{3(k + 1)} } }{ \frac{ {(-1)}^k(6k)! }{ (3k)!{(k!)}^{3}640320^{3k} } }                                                                                           \\
                          & = \frac{{(-1)}^{k + 1}(6(k + 1))!}{(3(k + 1))!{((k + 1)!)}^{3} 640320^{3(k + 1)}}  \cdot \frac{(3k)!{(k!)}^{3}640320^{3k}}{{(-1)}^k(6k)!}                                                                                                          \\
                          & = \frac{-1 \cdot {(-1)}^{k} (6k + 6)! }{ (3k + 3)! {((k + 1) \cdot k!)}^{3} 640320^{3k + 3}}  \cdot \frac{(3k)!{(k!)}^{3}640320^{3k}}{{(-1)}^k(6k)!}                                                                                               \\
                          & = -\frac{\Ccancel[red]{{(-1)}^{k}} (6k + 6)!\ (3k)! \Ccancel[Orange]{{(k!)}^{3}} \Ccancel[Magenta]{640320^{3k}} }{ (3k + 3)! {(k + 1)}^{3} \Ccancel[Orange]{{(k!)}^{3}} \Ccancel[Magenta]{640320^{3k}} 640320^{3} \ \Ccancel[red]{{(-1)}^k} (6k)!} \\
                          & = -\frac{(6k + 6)! (3k)! }{ (3k + 3)! {(k + 1)}^{3} 640320^{3} (6k)!}                                                                                                                                                                              \\
                          & = -\frac{(6k + 6)(6k + 5)(6k + 4)(6k + 3)(6k + 2)(6k + 1)\Ccancel[red]{(6k)!} \Ccancel[Orange]{(3k)!} }{ (3k + 3)(3k + 2)(3k + 1)\Ccancel[Orange]{(3k)!} {(k + 1)}^{3} 640320^{3} \Ccancel[red]{(6k)!}}                                            \\
                          & = -\frac{8\Ccancel[red]{(3k + 3)}(6k + 5)\Ccancel[Orange]{(3k + 2)}(6k + 3)\Ccancel[Magenta]{(3k + 1)}(6k + 1)}{ \Ccancel[red]{(3k + 3)}\Ccancel[Orange]{(3k + 2)}\Ccancel[Magenta]{(3k + 1)} {(k + 1)}^{3} 640320^{3}}                            \\
                          & = -\frac{24(6k + 5)(2k + 1)(6k + 1)}{{(k + 1)}^{3} 640320^{3}}                                                                                                                                                                                     \\
  \end{align*}
}

\end{document}

